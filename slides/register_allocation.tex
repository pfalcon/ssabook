  
%%%%%%%%%%%%%%%%%%%%%%%%%%%%%%%%%%%%%%%%%%%%%%%%
%%%% Register allocation
%%%%%%%%%%%%%%%%%%%%%%%%%%%%%%%%%%%%%%%%%%%%%%%%
\section{Register Allocation (F. Bouchez Tichadou)}

\subsection*{The rules of the game}
\begin{frame}
  \frametitle{What is register allocation?}
    \begin{block}{Assign variables (unbounded) to:}
       Registers (\includegraphics[width=0.12\textwidth]{registers.fig}) \hskip1em or \hskip1em
       memory (infinite)
    \end{block}
   \begin{block}{Architectural subtleties}
        Specific registers (sp, fp, r0), variable affinities (auto-inc), register pairing (64 bits
          ops), distributed register banks, etc.
    \end{block}
   \begin{minipage}{0.7\textwidth}
    \begin{block}{Rules of the game}
      \begin{itemize}
        \item Fixed instruction schedule
        \item \alert{Spill}: insert {\sc loads} and {\sc stores}
        \item \alert{Coalesce}: delete register-to-register {\sc moves}
        \item \alert{Split}: add register-to-register {\sc moves}
      \end{itemize}
    \end{block}
   \end{minipage}
\end{frame}

\begin{frame}
  \frametitle{What is register allocation (cont'd)?}
  \begin{block}{Allocation versus assignment}
    \begin{itemize}
      \item Allocation: which variables are allocated to memory (spilled), which ones are allocated to registers?
      \item Assignment: assign non spilled variables to registers
    \end{itemize}
  \end{block}
  \begin{block}{Allocation problem}
    \begin{itemize}
      \item \alert<3>{The spill test:} is spilling necessary (assignment feasible)?
      \item \alert<2>{What to spill and where to insert loads \& stores?}
    \end{itemize}
  \end{block}
\end{frame}

\begin{frame}
\frametitle{In-Two-Phases Register Allocation and Tree-Scan}

\only<1>{\tikzstyle{spilltest} = [concept]}
\only<2->{\tikzstyle{spilltest} = [concept]}
\only<1>{\tikzstyle{liverangesplitting} = [concept, opacity=0]}
\only<2->{\tikzstyle{liverangesplitting} = [concept]}
\only<1-2>{\tikzstyle{twophases} = [concept, opacity=0]}
\only<3->{\tikzstyle{twophases} = [concept]}
\only<1-3>{\tikzstyle{treescan} = [concept, opacity=0]}
\only<4->{\tikzstyle{treescan} = [concept]}
\only<1-3>{\tikzstyle{SSAproperties} = [concept, opacity=0]}
\only<4->{\tikzstyle{SSAproperties} = [concept]}
\tikzstyle{child6} = [concept color=blue,opacity=0]
\tikzstyle{child7} = [concept color=blue,opacity=0]
\tikzset{mybox/.style = {rectangle, minimum width=0pt, minimum height=0pt, text
    = black, font = \small}}

\begin{tikzpicture}[transform shape,scale=0.9]
    \path[mindmap,concept color=gray,text=white]
    node[spilltest, scale=0.7] (SPILL) at (0,-3.5) {
      {\bf Spill test:}\\
      \only<1>{(NP-complete?}
      \only<2->{$r$-colorable\\ $\Longleftrightarrow$\\ $\textrm{MAXLIVE}\leq r$} 
    }

    child[liverangesplitting, grow=90, level distance=100] { node[concept, scale=0.7] (SPLIT) {Live-range splitting} }

    node[twophases, scale=0.7] (TWO) at (4,-4.5) {Two-phases reg-alloc\\ Simpler/better}
    node[treescan, scale=0.7] (TREESCAN) at (9,-5) {Tree-scan\\ Faster/less memory}
    child[SSAproperties, grow=120, level distance=160] { node[concept, scale=0.7] (SSA)  {SSA properties} };

    \draw<3-> [circle connection bar] (SPILL) edge (TWO);
    \draw<4-> [circle connection bar] (TWO) edge (TREESCAN);
\end{tikzpicture}

\begin{block}{}
  \only<1>{Chaitin's reduction proves the NP-completeness of the spill test...}%
  \only<2>{...under the assumption that live-range splitting is not possible.}%
  \only<3>{Spill test should not be coupled to global coloration anymore}%
  \only<4>{Tree-shaped SSA live-ranges allows for tree-based allocation schemes} 

\end{block}
\end{frame}

\subsection*{How SSA simplifies register allocation}

%BB
\begin{frame}<handout:0>
  \frametitle{``Spilling easier on a BB than on a general CFG''}
    \begin{columns}
    \column{0.3\textwidth}{%BB large
      \begin{center}
	\blue{Basic block}\\
	\only<1>{\includegraphics[height=5cm]{code_BB-1.fig}}%
	\only<2>{\includegraphics[height=5cm]{code_BB-2.fig}}%
	\only<3>{\includegraphics[height=5cm]{code_BB-3.fig}}%
	\only<4>{\includegraphics[height=5cm]{code_BB-4.fig}}%
	\only<5>{\includegraphics[height=5cm]{code_BB-5.fig}}%
	\only<6>{\includegraphics[height=5cm]{code_BB-6.fig}}%
	\only<7>{\includegraphics[height=5cm]{code_BB-7.fig}}%
	\only<8>{\includegraphics[height=5cm]{code_BB-8.fig}}%
	\only<9>{\includegraphics[height=5cm]{code_BB-9.fig}}%
	\only<10>{\includegraphics[height=5cm]{code_BB-10.fig}}%
	\only<11>{\includegraphics[height=5cm]{code_BB-11.fig}}%
	\only<12>{\includegraphics[height=5cm]{code_BB-12.fig}}%
	\only<13>{\includegraphics[height=5cm]{code_BB-13.fig}}%
	\only<14>{\includegraphics[height=5cm]{code_BB-14.fig}}%
	\begin{block}{}
	  \begin{itemize}
	  \item $\textsf{MAXLIVE}\leq r$
	  \item Linear scan
	  \end{itemize}
	\end{block}
      \end{center}
    }
    \column{0.3\textwidth}{%SSA small
      \begin{center}
	\blue{SSA code}\\
	\includegraphics[height=2.5cm]{code_SSA-1.fig}%
      \end{center}
    }
    \column{0.3\textwidth}{%CFG small
      \begin{center}
	\blue{General CFG}\\
	\includegraphics[height=2.5cm]{code_CFG-1.fig}    
      \end{center}
    }
    \end{columns}
\end{frame}
 
%CFG
\begin{frame}<handout:0>
  \frametitle{``Spilling easier on a BB than on a general CFG''}
  \begin{columns}
    \column{0.15\textwidth}{%BB
      \begin{center}
	\blue{BB}\\
	\includegraphics[height=2.5cm]{code_BB-8.fig}
      \end{center}
    }
    \column{0.15\textwidth}{%SSA small
       \begin{center}
	 \blue{SSA}\\
	 \includegraphics[height=2.5cm]{code_SSA-1.fig}%
       \end{center}
    }
    \column{0.45\textwidth}{%CFG
      \begin{center}
	\blue{General control flow graph}\\
	\includegraphics[height=5cm]{code_CFG_static.fig}    
	\includegraphics[height=3cm]{IntGraph-stuck.fig}  
	\begin{alertblock}{}
	  \begin{itemize}
	  \item Coloring test
	  \item Greedy coloring
	  \end{itemize}
	\end{alertblock}
      \end{center}
   }%
  \end{columns}
\end{frame}

%SSA
\begin{frame}<handout:0>
  \frametitle{``Under SSA: the dominance tree''}
    \begin{columns}
    \column{0.15\textwidth}{%
      \begin{center}
	\blue{BB}\\
	\includegraphics[height=2.5cm]{code_BB-8.fig}
      \end{center}
    }
    \column{0.55\textwidth}{%
      \begin{center}
	\blue{Static single assignment form}\\
	\only<1>{\includegraphics[height=5cm]{code_SSA-1.fig}}%
	\only<2>{\includegraphics[height=5cm]{code_SSA-2.fig}}%
	\only<3>{\includegraphics[height=5cm]{code_SSA-3.fig}}%
	\only<4>{\includegraphics[height=5cm]{code_SSA-4.fig}}%
	\only<5>{\includegraphics[height=5cm]{code_SSA-5.fig}}%
	\only<6>{\includegraphics[height=5cm]{code_SSA-6.fig}}%
	\only<7>{\includegraphics[height=5cm]{code_SSA-7.fig}}%
	\only<8>{\includegraphics[height=5cm]{code_SSA-8.fig}}%
	\only<9>{\includegraphics[height=5cm]{code_SSA-9.fig}}%
	\only<10>{\includegraphics[height=5cm]{code_SSA-10.fig}}%
	\only<11>{\includegraphics[height=5cm]{code_SSA-11.fig}}%
	\only<12>{\includegraphics[height=5cm]{code_SSA-12.fig}}%
	\only<13>{\includegraphics[height=5cm]{code_SSA-13.fig}}%
	\only<14>{\includegraphics[height=5cm]{code_SSA-14.fig}}%
	\begin{block}{}
	  \begin{itemize}
	  \item $\textsf{MAXLIVE}\leq r$
	  \item Tree scan
	  \end{itemize}
	\end{block}
      \end{center}
    }%
    \column{0.3\textwidth}{%
      \begin{center}
      \blue{CFG}\\
      \includegraphics[height=2.5cm]{code_CFG_static.fig}    
      \includegraphics[height=1.5cm]{IntGraph.fig}  
      \end{center}
    }     
  \end{columns}
\end{frame}

\begin{frame}
  \frametitle{``Under SSA: the dominance tree''}
    \begin{columns}
    \column{0.28\textwidth}{%BB
      \blue{Basic block}\\
      \includegraphics[height=3.5cm]{code_BB-8.fig}
      \begin{block}{}
	\begin{itemize}
	  \item $\textsf{MAXLIVE}\leq r$
	  \item Linear scan
	\end{itemize}
      \end{block}
    }
    \column{0.28\textwidth}{%SSA
      \blue{SSA form}\\
      \includegraphics[height=3.5cm]{code_SSA-8.fig}%
    \begin{block}{}
	 \begin{itemize}
	 \item $\textsf{MAXLIVE}\leq r$
	 \item Tree scan
	 \end{itemize}
      \end{block}
    }%
    \column{0.35\textwidth}{%CFG
      \blue{General CFG}\\
      \includegraphics[height=3.5cm]{code_CFG_static.fig}    
      \includegraphics[height=2cm]{IntGraph.fig}  
      \begin{alertblock}{}
	\begin{itemize}
	\item Coloring test
	\item Greedy coloring
	\end{itemize}
      \end{alertblock}
    }     
  \end{columns}
\end{frame}

\begin{frame}
  \frametitle{Why coalescing?}
  \begin{block}{Goal of coalescing}
    Removing the register-to-register copies [\texttt{move $a \gets b$}]
  \end{block}
   \begin{columns}
    \column{.38\textwidth}{
    Numerous copies due to:
    \smallskip
  \begin{itemize}[<+-| alert@+>]
    \item live-range splitting to avoid spilling % in particual, reg alloc in 2 phases
    \item register constraints
    \item \alert<.-.(1)>{SSA destruction}
  \end{itemize}
  \vfill
~
\vspace{6em}\\
  }%
 \hfill
 \setcounter{beamerpauses}{2}
  \column{.6\textwidth}{%
  \centering%
%  \begin{example}<+->%
  \only<+->{
  \begin{exampleblock}{}
    \begin{overprint}
    \onslide<.>%
    \tikzfigure{move-reg-constr}%
    %
    \onslide<+->%
    \vspace{-4mm}
    \tikzfigure{move-out-SSA}%
    %
    \end{overprint}%
  \end{exampleblock}%
  }%
 }
 \end{columns}%
\end{frame}

\subsubsection*{What is new?}

% Let us summaries what I have understood. With enough live-range splitting, there is no need to
% spill more than lowering MAXLIVE to the number of available registers. SSA based splitting do the job.
% But these ideas are not new!
\begin{frame}[label=past]
\frametitle{The past}
Live-range splitting already considered in the past:
\begin{itemize}
\item \emph{\textsf{MAXLIVE} registers are sufficient if aggressive splitting is performed}: already pointed out by Fabri, and Cytron \& Ferrante (but did not mention the possible need of critical edge splitting).
 \item Briggs tried SSA-based splitting (PhD thesis)
\end{itemize}
\uncover<2->{
But,
\begin{itemize}
 \item[\sad] These ideas were not exploited (coalescing not good enough?)
 \item[\red{\txtimpl}] Instead: sophisticated algorithms that split on demand
when the coloring fails, e.g., live-range splitting (Cooper and
Simpson), optimistic (Park and
Moon), priority based (Chow and
Hennessy, etc.
\end{itemize}}
\end{frame}

\begin{frame}[label=past]
  \frametitle{The present}
\includegraphics[width=\textwidth]{scheme_anim-4.fig}
  \begin{itemize}
  \item[1.] \alert{Spill} so as to lower register pressure to $\leq$ \#registers
  \item \alert{Split} so that interference graph is \gr{k} 
    ( $\sim$ $k$-colorable {\it \`a la} Chaitin)
  \item[2.] \alert{Color + Coalesce} to remove useless copies
  \end{itemize}

\begin{alertblock}{Exploit Greedy-$k$-colorable graph properties}
SSA based live-range splitting leads to an interference graph that can be $k$-colored ($k = \textsf{MAXLIVE}$) using Chaitin's simplification scheme. 
\end{alertblock}

\begin{alertblock}{Optimize coalescing separately from spilling}
\end{alertblock}
\end{frame}



\subsection*{Important notions}


\subsubsection*{Aggressive versus conservative coalescing}
\begin{frame}[label=past]
  \only<1-3|handout:1>{\frametitle{Aggressive coalescing}}
  \only<4-|handout:2>{\frametitle{Conservative coalescing}}
   \alert<1-3|handout>{Aggressive}
  coalescing may lead to spilling. \alert<4|handout>{Conservative} coalescing
  takes colorability into account.  

\vfill
  \begin{center}    \begin{tikzpicture}[scale=1.1, auto,swap]
\small

      % a simple graph
      \node[vertex] (b) at (1,1) {\only<-3>{b}\only<4>{\red{b}}} ;
      \node[vertex] (c) at (1,0) {c} ;
      \node[vertex] (a) at (0,1) {\only<2-3>{\red{a}}\only<1,4>{a}} ;
      \node[vertex] (d) at (0,0) {\only<1>{d}\only<2->{\red{d}}} ;
      \path[interf] (b) -- (c) ;
      \path[interf] (a) -- (b) ;
      \path[interf] (c) -- (d) ;
      \path[affinity] (a) -- (d) ;
      \path[affinity] (b) -- (d) ;

      \path node at (0,-1) {} ;

      \path<3-> node[draw,anchor=south west] at (0,-1) {2-colorable} ;

      % une petite fleche
      \draw[->,thick, double] (1.5,1.5) .. controls (2,2) and (3,2) .. (3.5,1.5) ;
      % avec du texte
      \path<-3> node[draw,anchor=south] at (2.5,2) {Coalesce aggressively} ;
      \path<4-> node[draw,anchor=south] at (2.5,2) {Coalesce conservatively} ;

      % on decale la suite vers la droite
      \begin{scope}[xshift=4cm]
        % invisible node so that it doesn't move when changing slides
        \path node at (1.3,1) {} ;

        % les memes qui n'ont pas bouge
        \path<2-> node[vertex] (b) at (1,1) {b} ;
        \path<2-> node[vertex] (c) at (1,0) {c} ;
        \path[interf]<2-> (c) -- (b) ;

        % coalescing de a et d
        \path[vertex]<2-3> node[vertex] (ad) at (0,0.5) {\red{ad}} ;
        \path[interf]<2-3> (ad) -- (b) ;
        \path[interf]<2-3> (ad) -- (c) ;
        \path[affinity]<2-3> (b) .. controls (0.45,1) .. (ad) ;

        \path<3> node[draw, anchor=south east,red] at (1,-1) {Not 2-colorable};

        % coalescing de b et d
        \path<4> node[vertex] (a) at (0,1) {a} ;
        \path<4> node[vertex](bd) at (1,1) {\red{bd}} ;
        \path[interf]<4> (a) -- (bd) ;
        \path[affinity]<4> (a) .. controls (0.4,0.7) and (0.6,0.7) .. (bd) ;

        \path<4> node[draw, anchor=south east,green] at (1,-1) {2-colorable} ;
      \end{scope}

    \end{tikzpicture}


\end{center}
\end{frame}

\subsubsection*{\Gr{k} graphs}
\begin{frame}[label=past]
  \frametitle{\Gr{k} graphs}
\vspace{-2cm}
$k$-colorability is hard to check, but
\alert{greedy-k-colorability} is easy.
  \begin{block}{}
  Check greedy-$k$-colorability: simplify nodes with $<k$ neighbors.% {\it ($k=3$)}
  \end{block}
  \vfill

%  \begin{overprint}
%  \begin{overlayarea}{\textwidth}{3cm}
  %    \setcounter{tmp}{0}
  
  \def\nodelist{0,10,9,7,6,13,5,4,12,2,3,1,11}
  \only<handout:0>{
    \def\simplifylist{0,10,9,7,6,13,5,4,12,2,3,1,11}
    \def\coloringlist{11,1,3,2,12,4,5,13,6,7,9,10,0}
  }
  \only<0|handout:1->{
    \def\simplifylist{0,10,9,7}
    \def\coloringlist{9,10,0}
  }
  
  \printaffinitystatusfalse

  \simplifyingtrue
  \foreach \simplifyme in \simplifylist {
    \only<+|handout:+>{\begin{tikzpicture}[overlay,shift={(current page.south west)}]

\ifx\inktikz\undefined
\else
%\begin{tikzpicture}
%\newif\ifsimplifying
%\simplifyingtrue
%\newif\ifprintaffinitystatus
%\printaffinitystatusfalse
  \def\nodelist{0,1,11,10,9,7,13,6,12,5,4,2,3}
  \def\simplifyme{9}
\fi


\begin{scope}[shift={(3cm,1cm)}]


\tikzstyle{normal interf} = [draw, black, thick]
\tikzstyle{simplified interf} = [draw, gray!30, thick]

\tikzstyle{affinity} = [dashed]

\tikzstyle{normal vertex} = [vertex]
\tikzstyle{simplified vertex} = [vsimpl]


  \countdef\counters=10

  \advance\counters by 1
  \countdef\returnvalue=\the\counters  % count for return value of functions

  \xdef\nodestatus{\the\counters} %nodestatus + node number
  \count1=\the\counters
  \advance\count1 by 20 
  \xdef\nodedegree{\the\count1} %nodestatus + node number

%  \message{Node status are from \nodestatus and degree from \nodedegree}

  \def\getanyval#1-#2.{{
    \count1=#1
    \advance\count1 by #2
    \global\returnvalue=\the\count\the\count1%
%    \message{Regs #1: got value of r#2 (i.e, reg {\the\count1}): it is \the\count\the\count1 (\the\returnvalue)}
  }}

  \def\setanyval#1-#2.#3{{
    \count1=#1
    \advance\count1 by #2%
    \global\count\the\count1=#3
  %  \message{Regs #1: set r#2 to #3 (i.e, reg {\the\count1} to \the\count\the\count1)}
  }}


  \def\getval#1{\getanyval\nodestatus-#1.}
  \def\setval#1.#2{\setanyval\nodestatus-#1.{#2}}

  \def\incrdegree#1{{
    \count1=\nodedegree
    \advance\count1 by #1
    \global\advance\count\the\count1 by 1
  }}

  \def\getdegree#1{\getanyval\nodedegree-#1.}


  \newif\ifsimplified

  \def\testedge#1-#2{
  \getval{#1}
    \ifnum\the\returnvalue=0
    \getval{#2}
      \ifnum\the\returnvalue=0
%        \message{NORMAL edge #1-#2}
        \simplifiedfalse
        \def\estyle{normal interf}
     \else
%       \message{SIMPLIFIED edge#1-#2}
       \simplifiedtrue
       \def\estyle{simplified interf}
     \fi
   \else
%     \message{SIMPLIFIED edge#1-#2}
     \simplifiedtrue
     \def\estyle{simplified interf}
   \fi
  }

  \def\testnode#1,#2.#3{%
    \getval{#1}%
    \ifnum\the\returnvalue=0
      \ifsimplifying
        \def\nstyle{normal vertex}
      \else
        \def\nstyle{vert#3}
      \fi
    \else
      \def\nstyle{simplified vertex}
%      \message{#1 (#2) seems to be simplified}
      \node at (stack#1) {$#2$};
    \fi
  }

  \def\newnode#1-#2-(#3)#4{
    \testnode#1,#2.#4
    \node[\nstyle] (n#1) at (#3) {$#2$};
  }

  \def\newedge#1-#2{
    \testedge{#1}-{#2}
    \ifsimplified\else
      \incrdegree{#1}
      \incrdegree{#2}
    \fi
    \path[\estyle] (n#1) -- (n#2);
  }

  \def\newaff#1-#2#3{
    \testedge{#1}-{#2}
    \ifprintaffinitystatus
      \path[affinity#3] (n#1) -- (n#2);
    \else\fi
    \path[\estyle,affinity] (n#1) -- (n#2);
  }


  \path (-1.8,-3.9) coordinate (stack top)
  ++(-2em,2.8) node[rotate=90] {{\sl Stack}};

  \def\stackheight{4.8}
  \draw[rounded corners=1mm] (stack top) ++(0,.5cm) -- ++(-1em,0) -- +(0,\stackheight) -- +(0,0) -- ++(2em,0) -- +(0,\stackheight);

  \def\addtostack#1{
%  \node[above=1ex of stack top,name=stack#1] (stack top) {};
    \node[above=.6em of stack top] (stack#1) {};
    \coordinate (stack top) at (stack#1);
%  \message{#1 put on stack}
  }
  

  \def\dosimplify#1{
  \foreach \x in \nodelist {
    \setval\x.1
    \addtostack{\x}
    \ifnum\x=#1
      \breakforeach
    \else
    \fi
  }}

  \foreach \x in \nodelist {
    \setval\x.0 ;
    \setanyval\nodedegree-\x.0 ;
  }
  \dosimplify{\simplifyme}

  \foreach \x in \nodelist {
    \getval\x
%    \message{***************************** \x  is \ifnum\the\returnvalue=0 normal\else simplified\fi}
  }

  \begin{scope}[scale=1.2,yshift=.5cm,xshift=-.5cm]

  \newnode1-a-(1,0)r
  \newnode2-b-(2,-2)g
  \newnode3-c-(1,-2)b
  \newnode4-d-(3,-1)r
  \newnode5-e-(4,-1)g
  \newnode6-f-(5,-1)r
  \newnode7-g-(6,-2)g

  \newnode9-i-(6,-1)b
  \newnode10-j-(7,0)g

  \newnode11-c'-(0,-2)g
  \newnode12-d'-(2,-1)r
  \newnode13-g'-(4,-2)b

  \newedge1-3
  \newedge1-2
  \newedge1-{11}
  \newedge2-4
  \newedge2-3
  \newedge2-{12}
  \newedge3-4
  \newedge3-{12}
  \newedge4-5
  \newedge5-6
  \newedge5-{13}
  \newedge6-7
  \newedge6-9
  \newedge6-{13}
  \newedge7-9
  \newedge9-{10}


  \newaff3-{11}r
  \newaff4-{12}g
  \newaff7-{13}r
  \newaff1-{10}r
\end{scope}

  % Now print the degree
  \ifsimplifying
  \foreach \x in \nodelist {
    \getdegree{\x}
%    \message{degree of \x is \the\returnvalue}
    \ifnum\the\returnvalue>0
    \adddegree\the\returnvalue to n\x.
    \else
    \fi

    %legend
	}
	{%[scale=.5,every node/.style={transform shape}]
	\it
	\scriptsize
    \node[degree label={red!80!magenta},right,transform shape,minimum size=1.5ex] (x) at (0,-2.8)  {} ;
	\node[right=1ex of x] {Hi-degree node};
    \node[degree label=green!80!blue,below=1ex of x,minimum size=1.5ex] (x) {};
	\node[right=1ex of x] {Low-degree node};
	}
  \else\fi



%\useasboundingbox (-2.1,-3.5) rectangle ++ (9.5,5.5);
%\dbbox


\end{scope}
\end{tikzpicture}
%
    \message{Simplifing \simplifyme}
    }
  }

  \simplifyingfalse
  \foreach \simplifyme in \coloringlist {
    \only<+|handout:+>{\begin{tikzpicture}[overlay,shift={(current page.south west)}]

\ifx\inktikz\undefined
\else
%\begin{tikzpicture}
%\newif\ifsimplifying
%\simplifyingtrue
%\newif\ifprintaffinitystatus
%\printaffinitystatusfalse
  \def\nodelist{0,1,11,10,9,7,13,6,12,5,4,2,3}
  \def\simplifyme{9}
\fi


\begin{scope}[shift={(3cm,1cm)}]


\tikzstyle{normal interf} = [draw, black, thick]
\tikzstyle{simplified interf} = [draw, gray!30, thick]

\tikzstyle{affinity} = [dashed]

\tikzstyle{normal vertex} = [vertex]
\tikzstyle{simplified vertex} = [vsimpl]


  \countdef\counters=10

  \advance\counters by 1
  \countdef\returnvalue=\the\counters  % count for return value of functions

  \xdef\nodestatus{\the\counters} %nodestatus + node number
  \count1=\the\counters
  \advance\count1 by 20 
  \xdef\nodedegree{\the\count1} %nodestatus + node number

%  \message{Node status are from \nodestatus and degree from \nodedegree}

  \def\getanyval#1-#2.{{
    \count1=#1
    \advance\count1 by #2
    \global\returnvalue=\the\count\the\count1%
%    \message{Regs #1: got value of r#2 (i.e, reg {\the\count1}): it is \the\count\the\count1 (\the\returnvalue)}
  }}

  \def\setanyval#1-#2.#3{{
    \count1=#1
    \advance\count1 by #2%
    \global\count\the\count1=#3
  %  \message{Regs #1: set r#2 to #3 (i.e, reg {\the\count1} to \the\count\the\count1)}
  }}


  \def\getval#1{\getanyval\nodestatus-#1.}
  \def\setval#1.#2{\setanyval\nodestatus-#1.{#2}}

  \def\incrdegree#1{{
    \count1=\nodedegree
    \advance\count1 by #1
    \global\advance\count\the\count1 by 1
  }}

  \def\getdegree#1{\getanyval\nodedegree-#1.}


  \newif\ifsimplified

  \def\testedge#1-#2{
  \getval{#1}
    \ifnum\the\returnvalue=0
    \getval{#2}
      \ifnum\the\returnvalue=0
%        \message{NORMAL edge #1-#2}
        \simplifiedfalse
        \def\estyle{normal interf}
     \else
%       \message{SIMPLIFIED edge#1-#2}
       \simplifiedtrue
       \def\estyle{simplified interf}
     \fi
   \else
%     \message{SIMPLIFIED edge#1-#2}
     \simplifiedtrue
     \def\estyle{simplified interf}
   \fi
  }

  \def\testnode#1,#2.#3{%
    \getval{#1}%
    \ifnum\the\returnvalue=0
      \ifsimplifying
        \def\nstyle{normal vertex}
      \else
        \def\nstyle{vert#3}
      \fi
    \else
      \def\nstyle{simplified vertex}
%      \message{#1 (#2) seems to be simplified}
      \node at (stack#1) {$#2$};
    \fi
  }

  \def\newnode#1-#2-(#3)#4{
    \testnode#1,#2.#4
    \node[\nstyle] (n#1) at (#3) {$#2$};
  }

  \def\newedge#1-#2{
    \testedge{#1}-{#2}
    \ifsimplified\else
      \incrdegree{#1}
      \incrdegree{#2}
    \fi
    \path[\estyle] (n#1) -- (n#2);
  }

  \def\newaff#1-#2#3{
    \testedge{#1}-{#2}
    \ifprintaffinitystatus
      \path[affinity#3] (n#1) -- (n#2);
    \else\fi
    \path[\estyle,affinity] (n#1) -- (n#2);
  }


  \path (-1.8,-3.9) coordinate (stack top)
  ++(-2em,2.8) node[rotate=90] {{\sl Stack}};

  \def\stackheight{4.8}
  \draw[rounded corners=1mm] (stack top) ++(0,.5cm) -- ++(-1em,0) -- +(0,\stackheight) -- +(0,0) -- ++(2em,0) -- +(0,\stackheight);

  \def\addtostack#1{
%  \node[above=1ex of stack top,name=stack#1] (stack top) {};
    \node[above=.6em of stack top] (stack#1) {};
    \coordinate (stack top) at (stack#1);
%  \message{#1 put on stack}
  }
  

  \def\dosimplify#1{
  \foreach \x in \nodelist {
    \setval\x.1
    \addtostack{\x}
    \ifnum\x=#1
      \breakforeach
    \else
    \fi
  }}

  \foreach \x in \nodelist {
    \setval\x.0 ;
    \setanyval\nodedegree-\x.0 ;
  }
  \dosimplify{\simplifyme}

  \foreach \x in \nodelist {
    \getval\x
%    \message{***************************** \x  is \ifnum\the\returnvalue=0 normal\else simplified\fi}
  }

  \begin{scope}[scale=1.2,yshift=.5cm,xshift=-.5cm]

  \newnode1-a-(1,0)r
  \newnode2-b-(2,-2)g
  \newnode3-c-(1,-2)b
  \newnode4-d-(3,-1)r
  \newnode5-e-(4,-1)g
  \newnode6-f-(5,-1)r
  \newnode7-g-(6,-2)g

  \newnode9-i-(6,-1)b
  \newnode10-j-(7,0)g

  \newnode11-c'-(0,-2)g
  \newnode12-d'-(2,-1)r
  \newnode13-g'-(4,-2)b

  \newedge1-3
  \newedge1-2
  \newedge1-{11}
  \newedge2-4
  \newedge2-3
  \newedge2-{12}
  \newedge3-4
  \newedge3-{12}
  \newedge4-5
  \newedge5-6
  \newedge5-{13}
  \newedge6-7
  \newedge6-9
  \newedge6-{13}
  \newedge7-9
  \newedge9-{10}


  \newaff3-{11}r
  \newaff4-{12}g
  \newaff7-{13}r
  \newaff1-{10}r
\end{scope}

  % Now print the degree
  \ifsimplifying
  \foreach \x in \nodelist {
    \getdegree{\x}
%    \message{degree of \x is \the\returnvalue}
    \ifnum\the\returnvalue>0
    \adddegree\the\returnvalue to n\x.
    \else
    \fi

    %legend
	}
	{%[scale=.5,every node/.style={transform shape}]
	\it
	\scriptsize
    \node[degree label={red!80!magenta},right,transform shape,minimum size=1.5ex] (x) at (0,-2.8)  {} ;
	\node[right=1ex of x] {Hi-degree node};
    \node[degree label=green!80!blue,below=1ex of x,minimum size=1.5ex] (x) {};
	\node[right=1ex of x] {Low-degree node};
	}
  \else\fi



%\useasboundingbox (-2.1,-3.5) rectangle ++ (9.5,5.5);
%\dbbox


\end{scope}
\end{tikzpicture}
%
    \message{Coloring \simplifyme}%
    }
  }
  {
    \printaffinitystatustrue
    \def\simplifyme{0}
    \only<+|handout:+>{\begin{tikzpicture}[overlay,shift={(current page.south west)}]

\ifx\inktikz\undefined
\else
%\begin{tikzpicture}
%\newif\ifsimplifying
%\simplifyingtrue
%\newif\ifprintaffinitystatus
%\printaffinitystatusfalse
  \def\nodelist{0,1,11,10,9,7,13,6,12,5,4,2,3}
  \def\simplifyme{9}
\fi


\begin{scope}[shift={(3cm,1cm)}]


\tikzstyle{normal interf} = [draw, black, thick]
\tikzstyle{simplified interf} = [draw, gray!30, thick]

\tikzstyle{affinity} = [dashed]

\tikzstyle{normal vertex} = [vertex]
\tikzstyle{simplified vertex} = [vsimpl]


  \countdef\counters=10

  \advance\counters by 1
  \countdef\returnvalue=\the\counters  % count for return value of functions

  \xdef\nodestatus{\the\counters} %nodestatus + node number
  \count1=\the\counters
  \advance\count1 by 20 
  \xdef\nodedegree{\the\count1} %nodestatus + node number

%  \message{Node status are from \nodestatus and degree from \nodedegree}

  \def\getanyval#1-#2.{{
    \count1=#1
    \advance\count1 by #2
    \global\returnvalue=\the\count\the\count1%
%    \message{Regs #1: got value of r#2 (i.e, reg {\the\count1}): it is \the\count\the\count1 (\the\returnvalue)}
  }}

  \def\setanyval#1-#2.#3{{
    \count1=#1
    \advance\count1 by #2%
    \global\count\the\count1=#3
  %  \message{Regs #1: set r#2 to #3 (i.e, reg {\the\count1} to \the\count\the\count1)}
  }}


  \def\getval#1{\getanyval\nodestatus-#1.}
  \def\setval#1.#2{\setanyval\nodestatus-#1.{#2}}

  \def\incrdegree#1{{
    \count1=\nodedegree
    \advance\count1 by #1
    \global\advance\count\the\count1 by 1
  }}

  \def\getdegree#1{\getanyval\nodedegree-#1.}


  \newif\ifsimplified

  \def\testedge#1-#2{
  \getval{#1}
    \ifnum\the\returnvalue=0
    \getval{#2}
      \ifnum\the\returnvalue=0
%        \message{NORMAL edge #1-#2}
        \simplifiedfalse
        \def\estyle{normal interf}
     \else
%       \message{SIMPLIFIED edge#1-#2}
       \simplifiedtrue
       \def\estyle{simplified interf}
     \fi
   \else
%     \message{SIMPLIFIED edge#1-#2}
     \simplifiedtrue
     \def\estyle{simplified interf}
   \fi
  }

  \def\testnode#1,#2.#3{%
    \getval{#1}%
    \ifnum\the\returnvalue=0
      \ifsimplifying
        \def\nstyle{normal vertex}
      \else
        \def\nstyle{vert#3}
      \fi
    \else
      \def\nstyle{simplified vertex}
%      \message{#1 (#2) seems to be simplified}
      \node at (stack#1) {$#2$};
    \fi
  }

  \def\newnode#1-#2-(#3)#4{
    \testnode#1,#2.#4
    \node[\nstyle] (n#1) at (#3) {$#2$};
  }

  \def\newedge#1-#2{
    \testedge{#1}-{#2}
    \ifsimplified\else
      \incrdegree{#1}
      \incrdegree{#2}
    \fi
    \path[\estyle] (n#1) -- (n#2);
  }

  \def\newaff#1-#2#3{
    \testedge{#1}-{#2}
    \ifprintaffinitystatus
      \path[affinity#3] (n#1) -- (n#2);
    \else\fi
    \path[\estyle,affinity] (n#1) -- (n#2);
  }


  \path (-1.8,-3.9) coordinate (stack top)
  ++(-2em,2.8) node[rotate=90] {{\sl Stack}};

  \def\stackheight{4.8}
  \draw[rounded corners=1mm] (stack top) ++(0,.5cm) -- ++(-1em,0) -- +(0,\stackheight) -- +(0,0) -- ++(2em,0) -- +(0,\stackheight);

  \def\addtostack#1{
%  \node[above=1ex of stack top,name=stack#1] (stack top) {};
    \node[above=.6em of stack top] (stack#1) {};
    \coordinate (stack top) at (stack#1);
%  \message{#1 put on stack}
  }
  

  \def\dosimplify#1{
  \foreach \x in \nodelist {
    \setval\x.1
    \addtostack{\x}
    \ifnum\x=#1
      \breakforeach
    \else
    \fi
  }}

  \foreach \x in \nodelist {
    \setval\x.0 ;
    \setanyval\nodedegree-\x.0 ;
  }
  \dosimplify{\simplifyme}

  \foreach \x in \nodelist {
    \getval\x
%    \message{***************************** \x  is \ifnum\the\returnvalue=0 normal\else simplified\fi}
  }

  \begin{scope}[scale=1.2,yshift=.5cm,xshift=-.5cm]

  \newnode1-a-(1,0)r
  \newnode2-b-(2,-2)g
  \newnode3-c-(1,-2)b
  \newnode4-d-(3,-1)r
  \newnode5-e-(4,-1)g
  \newnode6-f-(5,-1)r
  \newnode7-g-(6,-2)g

  \newnode9-i-(6,-1)b
  \newnode10-j-(7,0)g

  \newnode11-c'-(0,-2)g
  \newnode12-d'-(2,-1)r
  \newnode13-g'-(4,-2)b

  \newedge1-3
  \newedge1-2
  \newedge1-{11}
  \newedge2-4
  \newedge2-3
  \newedge2-{12}
  \newedge3-4
  \newedge3-{12}
  \newedge4-5
  \newedge5-6
  \newedge5-{13}
  \newedge6-7
  \newedge6-9
  \newedge6-{13}
  \newedge7-9
  \newedge9-{10}


  \newaff3-{11}r
  \newaff4-{12}g
  \newaff7-{13}r
  \newaff1-{10}r
\end{scope}

  % Now print the degree
  \ifsimplifying
  \foreach \x in \nodelist {
    \getdegree{\x}
%    \message{degree of \x is \the\returnvalue}
    \ifnum\the\returnvalue>0
    \adddegree\the\returnvalue to n\x.
    \else
    \fi

    %legend
	}
	{%[scale=.5,every node/.style={transform shape}]
	\it
	\scriptsize
    \node[degree label={red!80!magenta},right,transform shape,minimum size=1.5ex] (x) at (0,-2.8)  {} ;
	\node[right=1ex of x] {Hi-degree node};
    \node[degree label=green!80!blue,below=1ex of x,minimum size=1.5ex] (x) {};
	\node[right=1ex of x] {Low-degree node};
	}
  \else\fi



%\useasboundingbox (-2.1,-3.5) rectangle ++ (9.5,5.5);
%\dbbox


\end{scope}
\end{tikzpicture}
}
  }
\end{frame}

\subsubsection*{Incremental conservative coalescing}
\begin{frame}[label=past]
  \frametitle{Incremental conservative coalescing}
  Finding the optimal subset of affinities is
    hard. 
  \begin{block}{}
    Algorithms do \alert{incremental conservative coalescing}.
  \end{block}
  \begin{tikzpicture}[scale=1, auto,swap]
  \small
  % First we draw the vertices
  \foreach \pos/\name in { {(-2,0)/a}, {(0,2)/b}, {(-1,-1)/c}, 
  {(0,1)/e}, {(1,-1)/h}, {(2,0)/i}, {(0,-1)/m}, {(0,0)/f}}
  \node[vertex] (\name) at \pos {\name};

  % Interferences
  \foreach \source/ \dest in {f/e, f/m, m/c, m/h, c/a, a/b, b/e, b/i, h/i}
  \path[interf] (\source) -- (\dest);

  % where to put messages
  \path node (message) at (4,1) {};

  % Try coal d with f
  \path<1-2,4-|handout:1,3-> node[vertex] (d) at (-1,0) {d} ;
  \path<2|handout:0>[selected edge] (d) -- (f) ; 
  \foreach \dest in {a, c, e}
  \path[interf]<1-2,4-|handout:1,3-> (d) -- (\dest);
  %      \path<1,3-> \node[vertex] at (f) {f} ;
  \path[affinity] (d) -- (f) ;

  \path<3|handout:2> node[vertex] (df) at (0,0) {\red{df}} ;
  \foreach \dest in {a, c, e}
  \path[interf]<3|handout:2> (df) -- (\dest);

   % error
  \path<3|handout:2> node[draw, anchor=west,red] at (message) {Not greedy-3-colorable};


  % Try coal g with f
  \path<1-5,7-|handout> node[vertex] (g) at (1,0) {g} ;
  \path<5|handout:0>[selected edge] (g) -- (f) ; 
  \foreach \dest in {e, h, i}
  \path[interf]<1-5,7-|handout> (g) -- (\dest);
  \path[affinity] (g) -- (f) ;

\only<handout:0>{
  \path<6> node[vertex] (fg) at (0,0) {fg} ;
  \foreach \dest in {e, h, i}
  \path[interf]<6> (fg) -- (\dest);

  \path<6> node[draw, anchor=west,red] at (message) {Not greedy-3-colorable};
}

  % Then try j with k
  \path<1-8|handout:-2> node[vertex] (j) at (3,2) {j} ;
  \path<1-8|handout:-3> node[vertex] (k) at (3,0) {k} ;	
  \path<8|handout:0>[selected edge] (j) -- (k) ; 

  \path[interf] (i) -- (k) ;
  \path[interf]<1-8|handout:-2> (b) -- (j) ;
  \path[affinity]<1-8|handout:-2> (k) -- (j) ;
  \path<9-|handout:3-> node[vertex] at (k) {\red{jk}} ;
  \path[interf]<9-|handout:3-> (b) -- (k) ;
  \path<9|handout:3> node[draw, anchor=west,green] at (4,0) {greedy-3-colorable};


  % Then jk with l
  \path<1-10|handout:-3> node[vertex] (l) at (3,-1) {l} ;
  \path<10|handout:0>[selected edge] (l) -- (k) ; 
  \path[affinity]<1-10|handout:-3> (k) -- (l) ;
  \path[interf]<1-10|handout:-3> (h) -- (l) ;
  \path[interf]<11-|handout:4> (h) -- (k) ;
  \path<11-|handout:4> node[vertex] at (k) {\red{jkl}} ;

  \path<11|handout:4> node[draw, anchor=west,green] at (4,0) {greedy-3-colorable};


\end{tikzpicture}



\vspace{2.5cm}

\end{frame}


\subsubsection*{Aggressive coalescing + decoalescing}

\begin{frame}[label=past]
  \frametitle{A gap}
  Incremental conservative is not optimal. 
  \begin{block}{}
  Greedy-$k$-colorable test might be stuck. Multiple node merging necessary to stay Greedy-$k$-colorable.
\end{block}
\only<1-4>{\begin{tikzpicture}[scale=1, auto,swap]
  \small
  % Tous les sommets qui ne bougent pas
  \foreach \pos/\name in { {(-2,0)/a}, {(0,2)/b}, {(-1,-1)/c}, 
  {(0,1)/e}, {(1,-1)/h}, {(2,0)/i}, {(3,2)/j}, 
  {(3,0)/k}, {(3,-1)/l}, {(0,-1)/m}}
  \node[vertex] (\name) at \pos {\name};

  \foreach \source/ \dest in
      {a/b, a/c, b/e, b/i, b/j, c/m, h/m, h/i, h/l, i/k}
      \path[interf] (\source) -- (\dest);

  % Les sommets qui changent
  \foreach \dpos / \fpos / \gpos / \date in 
       {{(-1,0)/(0,0)/(1,0)/1|handout:0}, {(0,0)/(0,0)/(1,0)/2-3|handout:0}, {(0,0)/(0,0)/(0,0)/4|handout:1}} 
     {        
       \path node (d) at \dpos {};
       \path node (f) at \fpos {};
       \path node (g) at \gpos {};
       
       \foreach \source/ \dest in %interferences
          {d/a, d/c, d/e, f/e, f/m, g/e, g/h, g/i}
          \path<\date>[interf] (\source) -- (\dest);
     }

  %top 1: d f g explosés
  \path<1|handout:0> node[vertex] (d) at (-1,0) {d};
  \path<1|handout:0> node[vertex] (f) at (0,0) {f};
  \path<1|handout:0> node[vertex] (g) at (1,0) {g};
  \path<1|handout:0>[selected edge] (d) -- (f);

  %top 2-3: only d & f coalesced
  \path<2-3|handout:0> node[vertr] at (0,0) {df};
  \path<2|handout:0> node[vertex] at (1,0) {g};
  \path<3|handout:0> node[vertr] at (1,0) {g};

  %top 4: dfg coalesced
  \path<4|handout:1> node[vertr] at (0,0) {dfg};

\end{tikzpicture}
 


\vspace{2.5cm}}%
% \only<5>{\begin{tikzpicture}[scale=1, auto,swap]
  \small
  % Draw a 7,11 network
  % First we draw the vertices
  \foreach \pos/\name in { {(-2,0)/a}, {(0,2)/b}, {(-1,-1)/c},
  {(0,1)/e}, {(0,0)/dfg}, {(1,-1)/h}, {(2,0)/i}, {(0,-1)/m}}
  \node[vertex] (\name) at \pos {\name};

  \foreach \pos/\name in {{(3,0)/jkl}}
  \path node (\name) at \pos {\name};

  \path<1-> node[vertex] at (jkl) {kl};

  \foreach \source/ \dest/ \start in
  {jkl/i/2, jkl/h/3, dfg/e/4, dfg/m/4, m/c/5, m/h/5, c/a/6, c/dfg/6,
  a/dfg/7, a/b/7, dfg/e/8, b/e/9, b/i/9, dfg/i/11, dfg/h/11, h/i/12}
  {
  \path[interf] (\source) -- (\dest);
  }

%  \path[interf]<0> (b) -- (jkl) ;

  \path<1-> node[vertex] (j) at (3,2) {j} ;
  \path[interf]<1-> (b) --(j) ;
  \path[affinity]<1-> (jkl) -- (j) ;

\end{tikzpicture}
 


\vspace{2.5cm}}%

\end{frame}
%
\begin{frame}[label=past]
  \frametitle{A gap}
  Incremental conservative is not optimal. 
  \begin{block}{}
  Greedy-$k$-colorable test might be stuck. Multiple node merging necessary to stay Greedy-$k$-colorable.
\end{block}
{\begin{tikzpicture}[scale=1, auto,swap]
  \small
  % Draw a 7,11 network
  % First we draw the vertices
  \foreach \pos/\name in { {(-2,0)/a}, {(0,2)/b}, {(-1,-1)/c},
  {(0,1)/e}, {(0,0)/dfg}, {(1,-1)/h}, {(2,0)/i}, {(0,-1)/m}}
  \node[vertex] (\name) at \pos {\name};

  \foreach \pos/\name in {{(3,0)/jkl}}
  \path node (\name) at \pos {\name};

  \path<1-> node[vertex] at (jkl) {kl};

  \foreach \source/ \dest/ \start in
  {jkl/i/2, jkl/h/3, dfg/e/4, dfg/m/4, m/c/5, m/h/5, c/a/6, c/dfg/6,
  a/dfg/7, a/b/7, dfg/e/8, b/e/9, b/i/9, dfg/i/11, dfg/h/11, h/i/12}
  {
  \path[interf] (\source) -- (\dest);
  }

%  \path[interf]<0> (b) -- (jkl) ;

  \path<1-> node[vertex] (j) at (3,2) {j} ;
  \path[interf]<1-> (b) --(j) ;
  \path[affinity]<1-> (jkl) -- (j) ;

\end{tikzpicture}
 


\vspace{2.5cm}}%
\end{frame}
%
\begin{frame}[label=past]
  \frametitle{Aggressive + decoalescing}
\begin{block}{}
\alert{Aggressive + de-coalescing} scheme: start from a completely aggressively coalesced graph, give up with some move until it gets Greedy-$k$-colorable again. 
\end{block}
  \begin{tikzpicture}[scale=1, auto,swap]
  \small
  % Tous les sommets qui ne bougent pas
  \foreach \pos/\name in { {(-2,0)/a}, {(0,2)/b}, {(-1,-1)/c},
  {(0,1)/e}, {(0,0)/dfg}, {(1,-1)/h}, {(2,0)/i}, {(0,-1)/m}}
  \node[vertex] (\name) at \pos {\name};

  \foreach \source/ \dest in
      {dfg/e, dfg/m, m/c, m/h, c/a, c/dfg,
        a/dfg, a/b, dfg/e, b/e, b/i, dfg/i, dfg/h, h/i}
      \path[interf] (\source) -- (\dest);

  % Les sommets qui changent
  \foreach \jpos / \kpos / \lpos / \date in 
       {{(3,0)/(3,0)/(3,0)/1}, {(3,2)/(3,0)/(3,0)/2}} 
     {        
       \path node (j) at \jpos {};
       \path node (k) at \kpos {};
       \path node (l) at \lpos {};
       
       \foreach \source/ \dest in %affinites
          {k/j, l/k}
          \path<\date>[affinity] (\source) -- (\dest);

       \foreach \source/ \dest in %interferences
          {k/i, l/h, b/j}
          \path<\date>[interf] (\source) -- (\dest);
     }

  %top 1: jkl fusionnes
  \path<1> node[vertex,fill=red!50] at (3,0) {jkl};

  %top 2: kl fusionnes et j seul
  \path<2> node[vertex] at (3,0) {kl};
  \path<2> node[vertex] at (3,2) {j};

\end{tikzpicture}
 


\vspace{2.5cm}
\end{frame}


\subsubsection*{Recoloring}
\newcommand{\cola}{red}
\newcommand{\colb}{blue}
\newcommand{\colc}{green}
\newcommand{\fulf}{black!50}
\tikzstyle ignode=[circle,fill,inner sep=2pt]
\tikzstyle cfgedge=[draw,thick,-stealth]
\tikzstyle cfgnode=[rectangle,draw,minimum height=0.7cm,minimum width=1.5cm]
\tikzstyle line=[cfgedge, shorten >= 1pt]
\tikzstyle block=[minimum width=1.5cm, minimum height=0.7cm, rectangle, draw={author in head/foot.bg}, thick, fill={date in head/foot.bg}, text centered, rounded corners]
\tikzstyle innerblock=[rectangle, draw=blue, thick, fill=blue!5, text centered, rounded corners]
\newcommand{\void}[1]{}
\newcommand<>{\igedge}[2]{\alt#3{\draw[thick,color=gray!20] (#1) -- (#2)}{\draw[thick,color=black] (#1) -- (#2)}}
\newcommand{\ignode}[2]{\fill[color=#2] (#1) circle(2.5pt)}
\newcommand<>{\ignodet}[2]{\temporal#3{\ignode{#1}{black}}{\ignode{#1}{gray!20}}{\ignode{#1}{#2}}}
%
\begin{frame}[label=past]
	\frametitle{Recoloring}
        \begin{block}{}
          \alert{Coalescing (coloring) + recoloring} scheme: start from a colored graph, try to assign move-related nodes the same color by recoloring its  neighborhood.
        \end{block}
	
	\begin{center}
		\begin{tikzpicture}[scale=0.8]
			\begin{scope}
				\tikzstyle{every node}=[circle,inner sep=2pt]
				\node<1-2|handout:1> [fill=\colb] (c) at (-2,2)    {};
				\node<3-|handout:2->[fill=\cola] (c) at (-2,2)    {};


				\node<5-|handout:3>[fill=\colb] (d) at (-2,3.25) {};
				\node<1-4|handout:-2>[fill=\cola] (d) at (-2,3.25) {};

				\node<7-|handout:0>[fill=\cola] (g) at (2,1)     {};
				\node<1-6|handout>[fill=\colb] (g) at (2,1)     {};

				\node<9-|handout:0>[fill=\colb] (e) at (1.5,2)   {};
				\node<1-8|handout>[fill=\cola] (e) at (1.5,2)   {};

				\node<11-|handout:0>[fill=\cola] (a) at (0.75,4)  {};
				\node<1-10|handout>[fill=\colb] (a) at (0.75,4)  {};

				\node[fill=\colc] (b) at (-0.75,4) {};
				\node[fill=\colb] (y) at (0,1.25)  {};
				\node[fill=\colc] (f) at (2,3.25)  {};
				\node[fill=\cola] (x) at (0,0)     {};

				\uncover<2-3|handout:-2>{\draw[thick] (c) circle (8pt);}
				\uncover<4-5|handout:3>{\draw[thick] (d) circle (8pt);}
				\uncover<6-7|handout:0>{\draw[thick] (g) circle (8pt);}
				\uncover<8-9|handout:0>{\draw[thick] (e) circle (8pt);}
				\uncover<10-11|handout:0>{\draw[thick] (a) circle (8pt);}
			\end{scope}
			\draw[thick] (b)--(d)--(c)--(b)--(a)--(e)--(b);
			\draw[thick] (g)--(e)--(f)--(a);
			\draw[thick] (x) -- (y);

			\draw<3-|handout:2->[thick,dashed,color=\fulf] (c) -- node[left]{{\small 2}} (x);
			\draw<1-2|handout:1>[thick,dashed] (c) -- node[left]{{\small 2}} (x);

			\draw[thick,dashed] (b) -- node[left]{{\small 2}} (y);

			\draw<7-|handout:0>[thick,dashed,color=\fulf] (x) -- node[below]{{\small 1}} (g);
			\draw<1-6|handout>[thick,dashed] (x) -- node[below]{{\small 1}} (g);
			
			\draw<9-|handout:0>[thick,dashed,color=\fulf] (y) -- node[below]{{\small 1}} (e);
			\draw<1-8|handout>[thick,dashed] (y) -- node[below]{{\small 1}} (e);
		\end{tikzpicture}
	\end{center}
\end{frame}



\subsubsection*{Different coalescing schemes}
\begin{frame}[label=past]
\frametitle{Different coalescing schemes}
\vfill
 \centering{
   \input{tikz/coal-all.ptikz}
 }
\end{frame}


\subsection*{Incremental coalescing. HOWTO}

\subsubsection*{From iterated-register coalescing to {\sf brute-force}}
\begin{frame}[label=past]
\frametitle{Iterated register coalescing}
\begin{block}{}
Iterated register coalescing uses an incremental scheme.
\end{block}
\only<1-|handout>{\includegraphics[width=\textwidth]{scheme_anim-4.fig}}

\only<1|handout:1>{\includegraphics[width=\textwidth]{scheme_iterated_anim-1.fig}}%
\only<2|handout:0>{\includegraphics[width=\textwidth]{scheme_iterated_anim-2.fig}}%
\only<3|handout:0>{\includegraphics[width=\textwidth]{scheme_chaitin_anim-1.fig}}%
\only<4|handout:0>{\includegraphics[width=\textwidth]{scheme_chaitin_anim-2.fig}}%
\only<5|handout:2>{\includegraphics[width=\textwidth]{scheme_chaitin_anim-3.fig}}%
\begin{itemize}
  \item<2-|handout:2>[\txtimpl] spilling and coalescing are not intermixed
  \item<5-|handout:2>[\txtimpl] if local coalescing tests (B/G) fail, run
  \textsf{brute-force} test
\end{itemize}
\end{frame}

\begin{frame}[label=rules]
  \frametitle{Incremental conservative local rules}
  \input{tikz/coalescing-rules-new.tikz}
\end{frame}

\begin{frame}[label=scheme]
\frametitle{The \textsf{brute-force} solution. HOWTO}
\only<1->{\includegraphics[width=\textwidth]{scheme_anim-4.fig}}%

\only<1|handout:0>{\includegraphics[width=\textwidth]{scheme_chaitin_anim-3.fig}}%
\only<2|handout:0>{\includegraphics[width=\textwidth]{scheme_chaitin_anim-4.fig}}%
\only<3->{\includegraphics[width=\textwidth]{scheme_chaitin_anim-5.fig}}%

\begin{itemize}
  \item[\txtimpl] spilling and coalescing are not intermixed
  \item[\txtimpl] if local coalescing tests (B/G) fail, run
  \textsf{brute-force} test
  \item<2->[\txtimpl] each affinity is considered only once
%TODO: expliquer les deux raisons qui motivent iterated
  \item<4>[\txtimpl] less work-lists to manipulate
\end{itemize}
\end{frame}

\subsubsection*{The {\sf brute-force} solution. Experimental results}

\begin{frame}[label=past]
\frametitle{Appel \& George's experiments}
From the standard ML new Jersey benchmark suite:
\begin{itemize}
 \item  Split everywhere. Compute and apply optimal spilling.
% 27 bb par region, max 1090. 231 instructions, max 8300. 6 registres
 \item  B/G: Iterated conservative coalescing (Briggs/George rules).
 \item[\txtimpl]  Coalescing challenge.
 \item  Optimistic: adapted optimistic (only for graphs with $degree < k$).
 \item Optimal: ILP-based solution (Grund and Hack).
\end{itemize}

\begin{center}
\begin{tikzpicture}
\begin{axis}[
    scale only axis,
    x tick label style={/pgf/number format/1000 sep=},
    ylabel=\#moves / \#opt,
    legend style={at={(0.5,-.25)},
        anchor=north,legend columns=-1},
    ybar stacked,
	x=2.5cm, y=1.5cm,
    bar width=30pt,
	axis y discontinuity=crunch,
	ymin=0.51,ymax=2.5,
	xmin=0.1,xmax=2.9,
	xtick={1,...,2},
	xticklabels={B/G,Optimistic},
	x tick label style={
 %   rotate=30,anchor=east
}
%	enlargelimits=false
]
\addplot coordinates
{
(1,2.225)
(2,1.355)
};

%\addplot coordinates {(1,0.1) (2,0.1) (3,0.1)};

%\legend{Remaining moves versus optimal (Grund and Hack~\cite{GruHa07})}
\end{axis}
\end{tikzpicture}

\end{center}
\end{frame}

\begin{frame}[label=past]
\frametitle{Experimental results. Quality}
\blue{Quality:}
\begin{center}
\begin{tikzpicture}
\begin{axis}[
    scale only axis,
    x tick label style={/pgf/number format/1000 sep=},
    ylabel=\#moves / \#opt,
    legend style={at={(0.5,-.25)},
        anchor=north,legend columns=-1},
     ybar stacked,
	x=2cm, y=2cm,
    bar width=30pt,
	axis y discontinuity=crunch,
	ymin=0.71,ymax=2.23,
	xmin=0.1,xmax=3.9,
	xtick={1,...,3},
	xticklabels={B/G,Optimistic,Brute-force},
	x tick label style={
 %   rotate=30,anchor=east
}
%	enlargelimits=false
]

\addplot coordinates
{(1,2.23)
(2,1.40) 
(3,1.14)}; 
\end{axis}
\end{tikzpicture}

\end{center}
\end{frame}

\begin{frame}[label=past]
\frametitle{Experimental results. Runtime}
\blue{Total time (for all graphs):}
\begin{itemize}
\item Briggs/George's test: $O(\frac{E}{V})$ 
\item \textsf{Brute-force} test: $O(E)$
\item In iterated scheme, if Briggs/George test fails: affinity inserted many times in the (ordered) work-list to be tested again.
\item In our simplified scheme with \textsf{brute-force} coalescing, if
Briggs/George test fails: only one \textsf{brute-force} test.
\end{itemize}
\begin{center}
\begin{tikzpicture}
\begin{axis}[
    x tick label style={/pgf/number format/1000 sep=},
    ylabel=seconds,
    legend style={at={(0.5,-.25)},
        anchor=north,legend columns=-1},
     ybar stacked,
	x=2cm, y=0.015cm,
    bar width=30pt,
	ymin=0,ymax=150,
	xmin=0.1,xmax=3.9,
	xtick={1,...,3},
        xticklabels={B/G,Optimistic,Brute-force},
	x tick label style={
 %   rotate=30,anchor=east
},
	enlargelimits=false
]
\addplot[fill=yellow!80!black] coordinates
{(1,79)
(2,45)
(3,135)
};

\end{axis}
\end{tikzpicture}

\end{center}
\end{frame}

\subsection*{Recoloring. HOWTO}
\begin{frame}[label=current]
	\frametitle{Recoloring}
        {\includegraphics[width=\textwidth]{recoloring.fig}}%
        \begin{center}
		\begin{tikzpicture}[scale=0.8]
			\begin{scope}
				\tikzstyle{every node}=[circle,inner sep=2pt]
				\node [fill=\colb] (c1) at (-2,2)    {};
				\node[fill=\cola] (d1) at (-2,3.25) {};
				\node[fill=\colb] (g1) at (2,1)     {};
				\node[fill=\cola] (e1) at (1.5,2)   {};
				\node[fill=\colb] (a1) at (0.75,4)  {};
				\node[fill=\colc] (b1) at (-0.75,4) {};
				\node[fill=\colb] (y1) at (0,1.25)  {};
				\node[fill=\colc] (f1) at (2,3.25)  {};
				\node[fill=\cola] (x1) at (0,0)     {};

                                \draw[thick] (b1)--(d1)--(c1)--(b1)--(a1)--(e1)--(b1);
                                \draw[thick] (g1)--(e1)--(f1)--(a1);
                                \draw[thick] (x1) -- (y1);
                                \draw[thick,dashed] (c1) -- node[left]{{\small 2}} (x1);
                                \draw[thick,dashed] (b1) -- node[left]{{\small 2}} (y1);
                                \draw[thick,dashed] (x1) -- node[below]{{\small 1}} (g1);
                                \draw[thick,dashed] (y1) -- node[below]{{\small 1}} (e1);

                                \draw[black,line width=5pt,->] (3,2) -- (5,2);

				\tikzstyle{every node}=[circle,inner sep=2pt]
				\node[fill=\cola] (c2) at (6,2)    {};
				\node[fill=\colb] (d2) at (6,3.25) {};
				\node[fill=\cola] (g2) at (10,1)     {};
				\node[fill=\colb] (e2) at (9.5,2)   {};
				\node[fill=\cola] (a2) at (8.75,4)  {};
				\node[fill=\colc] (b2) at (7.25,4) {};
				\node[fill=\colb] (y2) at (8,1.25)  {};
				\node[fill=\colc] (f2) at (10,3.25)  {};
				\node[fill=\cola] (x2) at (8,0)     {};

                                \draw[thick] (b2)--(d2)--(c2)--(b2)--(a2)--(e2)--(b2);
                                \draw[thick] (g2)--(e2)--(f2)--(a2);
                                \draw[thick] (x2) -- (y2);
                                
                                \draw[thick,dashed,color=\fulf] (c2) -- node[left]{{\small 2}} (x2);
                                \draw[thick,dashed] (b2) -- node[left]{{\small 2}} (y2);
                                \draw[thick,dashed,color=\fulf] (x2) -- node[below]{{\small 1}} (g2);
                                \draw[thick,dashed,color=\fulf] (y2) -- node[below]{{\small 1}} (e2);
			\end{scope}
		\end{tikzpicture}
              \end{center}
	\begin{itemize}
		\item \emph{Optimistically} try to assign move-related nodes the same color
		\item Resolve color clashes recursively through the graph
	\end{itemize}

\end{frame}

\begin{frame}[label=current]
	\frametitle{Recoloring}
\begin{block}{}
Conduct recoloring as a \emph{transaction}
\end{block}
	\begin{itemize}
	\item \emph{rollback} if color clash cannot be resolved.
	\item In a transaction:
          \begin{itemize}
          \item Do not recolor already recolored nodes to avoid recursion
          \item Look at all interference neighbors
          \item Try to find a color that is not used in the neighborhood
          \item If no such color is available: Pick the least used one
          \item If node is constrained: Set of available colors restricted
          \end{itemize}
        \end{itemize}
\end{frame}

\begin{frame}[label=current]
	\frametitle{Recoloring}
        \begin{block}{}
          Conduct transaction \emph{affinity component by affinity component}
        \end{block}
	\begin{center}
		\begin{tikzpicture}
%			\node<|handout:1> (a) at (0, 0) { };
%			\node<|handout:1> (d) at (6, 0) { };
%			\node<|handout:1> (e) at (8, 0) { };
%			\node<|handout:1> (f) at (10, 0) { };
			\fill<3|handout:1>[fill=black!20, rounded corners=4pt] ([shift={(-7mm,-7mm)}] a) rectangle ([shift={(7mm,7mm)}] d);
			\fill<3|handout:1>[fill=black!20, rounded corners=4pt] ([shift={(-7mm,-7mm)}] e) rectangle ([shift={(7mm,7mm)}] f);

			\node<3|handout:1> at ([yshift=1cm] a) {Chunk 1};
			\node<3|handout:1> at ([yshift=1cm] e) {Chunk 2};
			\node<1|handout:0>[style=ignode,label=below:$a$] (a) at (0, 0) { };
			\node<1|handout:0>[style=ignode,label=below:$b$] (b) at (2, 0) { };
			\node<1|handout:0>[style=ignode,label=below:$c$] (c) at (4, 0) { };
			\node<1|handout:0>[style=ignode,label=below:$d$] (d) at (6, 0) { };
			\node<1|handout:0>[style=ignode,label=below:$e$] (e) at (8, 0) { };
			\node<1|handout:0>[style=ignode,label=below:$f$] (f) at (10, 0) { };
			\node<2-|handout:1>[style=ignode,label=below:$a$,fill=\cola] (a) at (0, 0) { };
			\node<2-|handout:1>[style=ignode,label=below:$b$,fill=\cola] (b) at (2, 0) { };
			\node<2-|handout:1>[style=ignode,label=below:$c$,fill=\cola] (c) at (4, 0) { };
			\node<2-|handout:1>[style=ignode,label=below:$d$,fill=\cola] (d) at (6, 0) { };
			\node<2-|handout:1>[style=ignode,label=below:$e$,fill=\colb] (e) at (8, 0) { };
			\node<2|handout:0>[style=ignode,label=below:$f$,fill=\cola] (f) at (10, 0) { };
			\node<3-|handout:1>[style=ignode,label=below:$f$,fill=\colb] (f) at (10, 0) { };
			\draw[thick] (c) to[bend left=60] (e);
			\draw<1|handout:0>[dashed,thick] (a) --  (b);
			\draw<1|handout:0>[dashed,thick] (b) --  (c);
			\draw<1|handout:0>[dashed,thick] (c) --  (d);
			\draw<1-2|handout:0>[dashed,thick] (d) --  (e);
			\draw<1-2|handout:0>[dashed,thick] (e) --  (f);
			\draw<3-|handout:1>[dashed,thick,color=\fulf] (e) --  (f);


			\draw<2-|handout:1>[dashed,thick,color=\fulf] (a) --  (b);
			\draw<2-|handout:1>[dashed,thick,color=\fulf] (b) --  (c);
			\draw<2-|handout:1>[dashed,thick,color=\fulf] (c) --  (d);
		\end{tikzpicture}
	\end{center}
        \begin{itemize}
        \item Before recoloring: segregate components into interference-free chunks: already agree upon a set of lost affinities (aggressive coalescing)

        \item Try to satisfy affinities chunk by chunk
        \end{itemize}
\end{frame}


\begin{frame}[label=chunk]
	\frametitle{Recoloring a chunk}
	\begin{center}
		\begin{tikzpicture}
			\uncover<2->{
				\fill[fill=black!20, rounded corners=4pt] ([shift={(-4mm, -4mm)}] 2, 0) rectangle ([shift={(4mm, 14mm)}] 4, 0);
			}
			\node<1>[style=ignode] (n0) at (-1, 0) {};
			\node<1>[style=ignode] (n1) at (0, 0) {};
			\node<1>[style=ignode] (n2) at (1, 0) {};
			\node<1>[style=ignode] (n3) at (2, 0) {};
			\node<1>[style=ignode] (n4) at (3, 1) {};
			\node<1>[style=ignode] (n5) at (4, 0) {};

			\node<2->[style=ignode,fill=\cola] (n0) at (-1, 0) {};
			\node<2->[style=ignode,fill=\cola] (n1) at (0, 0) {};
			\node<2->[style=ignode,fill=\colb] (n2) at (1, 0) {};
			\node<2->[style=ignode,fill=\cola] (n3) at (2, 0) {};
			\node<2->[style=ignode,fill=\cola] (n4) at (3, 1) {};
			\node<2->[style=ignode,fill=\cola] (n5) at (4, 0) {};


			\draw<1>[dashed,thick] (n0) -- (n1);
			\draw<1>[dashed,thick] (n1) -- (n2);
			\draw<1>[dashed,thick] (n2) -- (n3);
			\draw<1>[dashed,thick] (n3) -- (n4);
			\draw<1>[dashed,thick] (n4) -- (n5);
			\draw<1>[dashed,thick] (n3) -- (n5);

			\draw<2->[dashed,thick,color=\fulf] (n0) -- (n1);
			\draw<2->[dashed,thick] (n1) -- (n2);
			\draw<2->[dashed,thick] (n2) -- (n3);
			\draw<2->[dashed,thick,color=\fulf] (n3) -- (n4);
			\draw<2->[dashed,thick,color=\fulf] (n4) -- (n5);
			\draw<2->[dashed,thick,color=\fulf] (n3) -- (n5);
			% \fans{n0}
			% \fans{n1}
			% \fans{n2}
			% \fans{n3}
			% \fans{n4}
			% \fans{n5}

			\uncover<5->{
				\draw[thick, rounded corners=4pt] 
				([shift={(-2mm,-6mm)}] n0.west) node[above=12mm, font=\scriptsize] {new chunk}
					rectangle ([shift={(2mm,6mm)}] n2.east);
			}
		\end{tikzpicture}
	\end{center}
	\begin{itemize}
		\item Define a sequence of colors to try for the chunk
		\item For each color $c$ in that sequence
			\begin{itemize}
				\item Try to recolor each node in the chunk to $c$
				\item<2-> Memorize the best sub-chunk in the chunk
			\end{itemize}
			\pause
		\item<3-> Select the color with the best sub-chunk
		\item<4-> Make the color of those nodes permanent
		\item<5-> Create new chunks from the rest and add them to the queue
	\end{itemize}
\end{frame}


\begin{frame}[label=current]
	\frametitle{Libfirm compiler~(\url{http://www.libfirm.org})}
	\begin{itemize}
		\item \text{CINT2000} benchmark suite
                  \begin{itemize}
                  \item Graph-based SSA IR
                  \item SSA-based register allocator
                  \end{itemize}
		\item Pentium 4 2.4GHz, 1GB RAM
		\item Affinity weights given by estimated execution frequencies
		\item Comparison to optimal solutions obtained from an ILP solver
	\end{itemize}
\end{frame}


\begin{frame}[label=current]
\frametitle{Experimental results. Quality}
\blue{Quality:}
\begin{center}
\begin{tikzpicture}
\begin{axis}[
    scale only axis,
    x tick label style={/pgf/number format/1000 sep=},
    ylabel=\#moves / \#opt,
    legend style={at={(0.5,-.25)},
        anchor=north,legend columns=-1},
     ybar stacked,
	x=2cm, y=2cm,
    bar width=30pt,
	axis y discontinuity=crunch,
	ymin=0.71,ymax=2.25,
	xmin=0.1,xmax=2.9,
	xtick={1,2},
	xticklabels={Iterated,Recoloring},
	x tick label style={
 %   rotate=30,anchor=east
}
%	enlargelimits=false
]

\addplot coordinates
{(1,2.25)
(2,1.25)}; 
\end{axis}
\end{tikzpicture}

\end{center}
\end{frame}

\begin{frame}
	\frametitle{Experimental results. Runtime}
	\begin{center}
		\begin{tikzpicture}[scale=1.2]
			\path (0,0)   node[anchor=south west] {\rotatebox{270}{\includegraphics[scale=0.35]{timehist.pdf}}};
			\draw (4,2)   node[anchor=west,font=\small]       {$c_1\cdot x$};
			\draw (4,3.5) node[anchor=south east,font=\small] {$c_2\cdot x^2$};
			\draw (6.8,5) node[anchor=south,font=\small]      {$c\cdot x^{1.2}$};
		\end{tikzpicture}
	\end{center}
\end{frame}


