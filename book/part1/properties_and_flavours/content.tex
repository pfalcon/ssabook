\chapter{Properties and flavours \Author{P. Brisk}}
\numberofpages{16}

\textbf{total page count for chapter: 16 pages}

\section{dominance}

\textbf{4 pages}

In this section, I will enumerate a number of
definitions and concepts that are fundamental
to control flow analysis.

The basic definitions are as follows:

\begin{enumerate}
\item Dominance
\item Strict Dominance
\item Immediate Dominator
\item Dominator Tree
\item Post-Dominance
\item Strict Post-Dominance
\item Immediate Post-Dominator
\item Post-Dominator Tree
\end{enumerate}

The more complex definitions, which are
required to understand SSA construction are
as follows:

\begin{enumerate}
\item Join Set
\item Dominance Frontier
\item Iterated Dominance Frontier
\item Split Set
\item Post-Dominance Frontier
\item Iterated Post Dominance Frontier
\end{enumerate}

I think that these concepts are fundamental,
and would benefit greatly from detailed
illustrations. I anticipate budgeting
two pages for text/definitions and two
pages for illustrations. 

\section{liveness}

\textbf{3 pages}

Basic definitions:
\begin{enumerate}
\item Definition of a variable
\item Use of a variable
\item Path in a CFG vs. Reachability
\item Formal definition of liveness at a point
\item Formal definition of the live range of a variable
\end{enumerate}

Briefly discuss/define LiveIn and LiveOut
sets. Give dataflow equations to compute
them, and refer back to Jeremy's discussion
of iterative dataflow analysis in the vanilla section.

Detailed illustration of a CFG with
variables defined and used. Each basic block
is augmented with LiveIn/LiveOut sets. 

I anticipate 2 pages of text and 1 page
for the illustration.

\section{interference graphs}

\textbf{2 pages}

Define variable live range intersection. 
Define variable interference (Chaitin's definition). 
Discuss why interference definition is intractable,
and give examples of cases where it becomes 
reasonable.

Define interference graph for a procedure.

Detailed example (hopefully, corresponding to
the procedure illustrated in the preceding 
section. 

Here, I anticipate 1 page of text and 1 page
for the illustration.

\section{SSA with dominance property}

\textbf{3 pages}

Formal definitions of SSA. Introduce
phi functions, and show why they are
necessary. Include an illustration of
a program before and after conversion to SSA.

Discuss strict vs. non-strict SSA conceptually.
Discuss how/why strict SSA implicitly assumes
that each variable has a pseudo-definition at
the CFG entry node. Without providing a full-blown
SSA construction algorithm, conceptually differentiate
reasoning regarding phi placement using dominance 
frontiers and join sets. 

\section{pruned/semi-pruned/minimal}

\textbf{2 pages}

Define the 3 flavors of SSA. "Borrow" the very nice
example from Briggs et al. SPE 1998 paper. 

Discuss "folklore" and patent regarding conversion
from minimal or semi-pruned SSA to pruned SSA via
dead code elimination. Provide an example if there
is space. 

\section{conventional}

\textbf{2 pages}

Introduce Sreedhar's definition of CSSA. Conceptually
explain how/why it simplifies SSA destruction, but do
not give algorithms. Borrow examples heavily from
Sreedhar's SAC 1999 paper.