\usepackage{makeidx}
\usepackage{fancyhdr}

\usepackage{progressbar}

\usepackage{graphicx}
\usepackage{subfigure}

\usepackage{datetime}

% Put the time in hh:mm format.
\settimeformat{xxivtime}

% Write latex source info (file, line) in dvi file
\usepackage{srcltx}

% Simplify handling of spaces after macros
\usepackage{xspace}

% Different colors to ease writing of comments & todos in text
\usepackage{xcolor}

% To comment large parts of text using \begin{comment}...\end{comment}
\usepackage{comment}

\usepackage{listings}
\usepackage{alltt}
\usepackage{amsmath, amsfonts, amssymb, amsxtra, amsopn, txfonts}
\usepackage{enumerate}
\usepackage[noend]{algorithmic}
\usepackage{algorithm}

% \usepackage{tikz}
% \usetikzlibrary{shapes,arrows,calc}

\usepackage{url}

% For hyperlinks in pdf
\usepackage[%pdftex, %draft,
  pdftitle={SSA-based Compiler Design},
  pdfauthor={Fabrice Rastello et al.},
  pdfsubject={Compiler Book},
  pdfkeywords={Compilers, Static Single Assignment},
  final,
  hyperindex=true,
  plainpages=false, pdfpagelabels, hypertexnames=false,
  % pdfview=Fit,  % << Never use this option or it will apply to all links (and screw them...)
  bookmarks, bookmarksopen, bookmarksnumbered=true, 
  %
  colorlinks=true,linkcolor=red!30!black, citecolor=blue!50!black, 
  urlcolor=blue!40!red,
  pdfborder={0 0 0},
  %
  bookmarksopenlevel=1,
  breaklinks,
  naturalnames% for algorithm2e
  ]{hyperref}
