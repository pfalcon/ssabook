\applynumberofpages\chapter{Hashed SSA form: HSSA \Author{M. Mantione}}
\numberofpages{11}


\section{Introduction}
\textbf{A few paragraphs}

Explain what ``HSSA'' means, its purpose, and the issues that must be solved to reach its goal.

\section{SSA and aliasing: $\mu$ and $\chi$ functions}
\textbf{1 page or less}

Briefly explain what aliasing is, and when it happens.
Then explain why aliasing is a problem when putting a program in SSA form.
Finally, introduce $\mu$ and $\chi$ functions as a way to keep the SSA representation correct.

\section{Introducing ``zero versions'' to limit the explosion of the number of variables}
\textbf{2 pages}

Explain why $\mu$ and $\chi$ functions generate too many variable versions, introduce the concept of ``zero version'', its formal definition, and the algorithm to detect zero versions of variables.

\section{SSA and indirect memory operations: indirect variables}
\textbf{1 page or more}

All of the above was still applies to ``regular'' local variables, not to arbitrary memory locations accessed indirectly.
Explain why it is desirable to also optimize indirect operations with a concrete example (or maybe more than one, for clarity? The simplest example is ``field access on structs in C and objects in object oriented languages in general'').
Introduce ``indirect variables'' and show how they work in SSA form.

\section{From \em{indirect} variables to \em{virtual} variables}
\textbf{2 pages}

Explain the problems with indirect variables, and the need for a sound way of dealing with aliasing among them (besides the straightforward one which is impractical).
Introduce and define virtual variables and how to apply aliasing and SSA to them.

\section{GVN and indirect memory operations: HSSA}
\textbf{1 page or less}

Briefly introduce GVN, and why it is desirable in this context.
Explain how the resulting IR (SSA form) is structured and how it works (the five kinds of nodes).

\section{Building HSSA}
\textbf{2 pages}

The HSSA construction algorithm.

\section{Using HSSA}
\textbf{1 page}

The effect that HSSA has on various optimizations.

\section{HSSA in the real world}
\textbf{half a page}

A concluding section, showing that HSSA is used in production compilers.
Maybe this should actually be moved to the introduction, and the chapter can just end with the ``Using HSSA'' section.

\chapterauthor{Mantione}


