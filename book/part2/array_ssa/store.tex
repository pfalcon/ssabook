In this section, we show how our Array SSA framework can be used to
identify redundant (dead) stores of array elements.  Dead store
elimination is related to load elimination, because scalar
replacement can convert non-redundant stores into redundant stores.
For example, consider the program in
Figure~\ref{fig:ex2}(a).  If it contained an additional
store of {\tt p.x} at the bottom, the first store of {\tt p.x} will
become redundant after scalar replacement.  The program after scalar
replacement will then be similar to the program shown in
Figure~\ref{fig:ex2}(c) as an example of dead
store elimination.

Our algorithm for dead store elimination is based on a backward
propagation of $\DEAD$ sets.  As in load elimination,
the propagation is {\it sparse} \ie\ it goes through $\phi$ nodes in
the Array SSA form rather than basic blocks in a control flow graph.
However, $u\phi$ functions are not used in
dead store elimination, since the ordering of uses is not relevant
to identifying a dead store.
Without $u\phi$ functions, it is possible for
multiple uses to access the same heap array name.
Hence, we use the notation $\langle A,s \rangle$
to refer to a specific use of
heap array $A$ in statement (instruction) $s$.
We also
use the term ``non-$\phi$'' to refer to any def or use that does not
appear in a (control or definition)
$\phi$ statement \ie\ a def or use from the original program.

Consider a $\phi$ {\it def} $A_i$, a $\phi$ or non-$\phi$ {\it use} $\langle A_j, s \rangle$, and a real
(non-$\phi$) def $A_k$ in Array SSA form.  We define the following
four sets:
\begin{eqnarray*}
\DEAD_{def}(A_i) & = & \{\; \V(x) \; | \; 
\mbox{element $x$ of array $A$ is dead at $\phi$ def $A_i$} \;\} \\
\DEAD_{use}(\langle A_j,s \rangle) & = & \{\; \V(x) \; | \; 
\mbox{element $x$ of array $A$ is dead at non-$\phi$ use of $A_j$ in statement $s$ } \;\} \\
\KILL(A_k) & = & \{\; \V(x) \;|\; \mbox{element $x$ of array $A$ is killed by
non-$\phi$ def of $A_k$} \;\}\\
\LIVE(A_i) & = & \{ \; \V(x) \;| \; \exists \; \mbox{a non-$\phi$ use $A_i[x]$ of
$\phi$ def $A_i$}\;\}
\end{eqnarray*}

The $\KILL$ and $\LIVE$ sets are local sets \ie\ they can be computed 
immediately
without propagation of data flow information.
If $A_i$ ``escapes'' from the procedure (\ie\ definition
$A_i$ is exposed on procedure exit), then we must conservatively
set $\LIVE(A_i) = \U^A_{ind}$, the universal set of index value numbers
for array $A$.  Note that in Java, every instruction that can potentially
throw an uncaught exception must be treated as a procedure exit, although this
property can be relaxed with some interprocedural analysis.

\begin{figure}
\begin{enumerate}
\item {\bf Propagation from the LHS to the RHS of a control $\phi$:}\\
Consider a control $\phi$ statement $s$ of the form, $A_2  := \phi(A_1,A_0)$.
In this case, the 
uses, $\langle A_1, s \rangle$ and $\langle A_0, s \rangle$, must both come from $\phi$ defs, and 
the propagation of $\DEAD_{def}(A_2)$ to the RHS is a simple copy
\ie\
$\DEAD_{use}(\langle A_1,s \rangle) = \DEAD_{def}(A_2)$ and
$\DEAD_{use}(\langle A_0,s \rangle) = \DEAD_{def}(A_2)$.

\item {\bf Propagation from the LHS to the RHS of a definition $\phi$:}\\
Consider a $d\phi$ statement $s$ of the form $A_2 := d\phi(A_1, A_0)$.
In this case use
$\langle A_1,s \rangle$ must come from a real (non-$\phi$) definition, and use $\langle A_0,s \rangle$  must
come from a $\phi$ definition.  The propagation
of $\DEAD_{def}(A_2)$ and $\KILL(A_1)$ to
$\DEAD_{use}(\langle A_0,s \rangle)$ is given by the equation,
$
\DEAD_{use}(\langle A_0,s \rangle) = \KILL(A_1) \cup \DEAD_{def}(A_2)
$.

\item {\bf Propagation to the LHS of a $\phi$ statement from uses
in other statements:}\\
Consider a definition or control $\phi$ statement of the form $A_i := \phi(\ldots)$.  The value of $\DEAD_{def}(A_i)$ is obtained by intersecting
the $\DEAD_{use}$
sets of all uses of
$A_i$, and subtracting out all value numbers that are not definitely different 
from every element of $\LIVE(A_i)$. 
This set is specified by the following equation:
$$
\DEAD_{def}(A_i) = \left(\bigcap_{\mbox{$s$ is a $\phi$ use of $A_i$}} \;\; \DEAD_{use}(\langle A_i,s \rangle)
\right) - \{ v | \exists w \in \LIVE(A_i) s.t. \neg DD(v,w) \}
$$
\end{enumerate}
\caption{Data flow equations for \protect{$\DEAD_{def}$} and 
\protect{$\DEAD_{use}$} sets}
\label{fig:dead}
\end{figure}


The data flow equations used to compute the $\DEAD_{def}$ and $\DEAD_{use}$ sets are given in Figure~\ref{fig:dead}.
The goal of our analysis is to find the maximal $\DEAD_{def}$ and $\DEAD_{use}$ sets that satisfy these
equations.  Hence our algorithm will initialize each 
$\DEAD_{def}$ and $\DEAD_{use}$
set to
$= \U^A_{ind}$
(for renamed arrays derived from original array $A$),
and then iterate on the equations till a fixpoint
is obtained.
\REM{
As in other data flow algorithms, termination is guaranteed by the monotonicity
of the transfer functions.  
Also, if the control flow graph for the input program is reducible, then the data flow analysis will iterate through back edges
of $\phi$ functions at most a constant number of times (assuming that the
maximum
loop nesting depth in the input program is constant).
}
After $\DEAD$ sets have been computed, we can determine if a real (non-$\phi$)
definition is redundant quite simply as follows.
Consider a real definition, $A_1[j] := \ldots$, followed by a definition
$\phi$ statement, $A_2 := d\phi(A_1,A_0)$.
Then, if $\V(j) \in \DEAD(A_2)$, then def (store) $A_1$ is redundant and can
be eliminated.
